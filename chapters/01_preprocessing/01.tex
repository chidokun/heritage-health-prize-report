\chapter{Data Preprocessing}

\section{Strategies}

Dataset được cung cấp chứa dữ liệu claim của bệnh nhân trong 3 năm (Y1, Y2 và Y3). Mục tiêu của bài toán là dùng dữ liệu của 3 năm này để dự đoán số ngày nằm viện của bệnh nhân đó trong năm thứ 4 (Y4).

Tuy nhiên, chúng ta không thể có dữ liệu của Y4 để xây dựng mô hình từ 3 năm đầu tiên do chính sách của Kaggle. Vì vậy, trong giới hạn của project này, chúng ta sẽ dùng dữ liệu Y1 và Y2 để làm train set và dùng Y3 để làm test set.

Dữ liệu hiện có chứa thông tin các lần claim trong năm của một bệnh nhân. Để có thể dự đoán số ngày nằm viện cho một bệnh nhân trong năm kế tiếp, dữ liệu claim cần được xử lý phù hợp để chuyển tất cả thông tin claim thành thông tin tổng hợp của một bệnh nhân.

Khi quan sát tập dữ liệu, ta nhận thấy có một số dữ liệu cố định của bệnh nhân, không phụ thuộc vào năm, ví dụ: "Sex", "AgeAtFirstClaim". Các thông tin này không thay đổi dù bệnh nhân có claim bao nhiêu lần đi chăng nữa. Cùng với đó là một số dữ liệu thay đổi theo mỗi lần claim, chẳng hạn: "PayDelay", "PrimaryConditionGroup", ...

Do đó, sẽ có 2 chiến lược xây dựng dữ liệu cho mô hình:

\textbf{Strategy A}: Chỉ dùng dữ liệu của 1 năm để dự đoán cho năm sau. Các thông tin trong bảng Claims sẽ được tổng hợp theo từng năm.

\textbf{Strategy B}: Kết hợp dữ liệu của tất cả các năm trong quá khứ để dự đoán cho năm tiếp theo. Dữ liệu claim của Y1 và Y2 trong bảng Claims sẽ được kết hợp lại.

Trong project này, ta sẽ xây dựng cả 2 mô hình để đánh giá chiến lược nào sẽ cho kết quả tốt hơn.

\section{Data Manipulation}

Phần này sẽ mô tả cách thức xử lý từng feature để tạo ra dataset cho 2 chiến lược ở trên.

\subsection{Bảng Members}

Bảng này chứa dữ liệu chung của bệnh nhân và không phụ thuộc thời gian.

\textbf{\underline{MemberID}}: Đã được de-identification, feature này chỉ được dùng làm index cho các feature khác và không dùng trong quá trình xây dựng mô hình.

\textbf{\underline{AgeAtFirstClaim}}: Biểu thị giá trị rời rạc cho khoảng độ tuổi và sẽ được chuyển thành dữ liệu số là giá trị trung bình của khoảng đô tuổi đó. Xem bảng sau:

\begin{table}[h!]
    \centering
    \begin{tabular}{||c c||} 
     \hline
     Giá trị ban đầu & Giá trị thay thế \\
     \hline\hline
     0-9 & 5 \\
     10-19 & 15 \\
     20-29 & 25 \\
     30-39 & 35 \\
     40-49 & 45 \\
     50-59 & 55 \\
     60-69 & 65 \\
     70-79 & 75 \\
     80+ & 85 \\
     NaN & -1 \\
     \hline
    \end{tabular}
    \caption{AgeAtFirstClaim}
    \label{table:1}
\end{table}

\textbf{\underline{Sex}}: Giới tính của bệnh nhân sẽ được chuyển thành one-hot vector. Gồm 3 giá trị: M, F và NaN

\subsection{Bảng Claims}

Chứa dữ liệu claim của bệnh nhân, mỗi feature trong bảng này sẽ được xử lý theo Y1, Y2 và kết hợp giữa Y1 và Y2 (Y12).

\textbf{\underline{ClaimCount}}: Đếm số lượng claim của mỗi bệnh nhân theo Y1, Y2 và Y12.

\textbf{\underline{Provider}}: Đếm số lượng Provider duy nhất của từng bệnh nhân theo Y1, Y2 và Y12.

\textbf{\underline{Vendor}}: Đếm số lượng Vendor duy nhất của từng bệnh nhân theo Y1, Y2 và Y12.

\textbf{\underline{PCP}}: Đếm số lượng PCP duy nhất của từng bệnh nhân theo Y1, Y2 và Y12.

\textbf{\underline{Specialty}}: Đếm số lượng mỗi giá trị của Speciaty cho từng bệnh nhân theo Y1, Y2 và Y12.

\textbf{\underline{PlaceSvc}}: Đếm số lượng mỗi giá trị của PlaceSvc cho từng bệnh nhân theo Y1, Y2 và Y12.

\textbf{\underline{PayDelay}}: Dữ liệu sẽ được chuyển thành kiểu số. Giá trị 162+ chuyển thành 162. Sau đó tính toán các giá trị thống kê như min, max, avg, std, sum của từng bệnh nhân theo Y1, Y2 và Y12.

\textbf{\underline{LengthOfStay}}: Dữ liệu được chuyển thành trung bình số ngày của mỗi giá trị và tính toán các giá trị thống kê như min, max, avg, std, sum của từng bệnh nhân theo Y1, Y2 và Y12. Xem bảng sau:

\begin{table}[h!]
    \centering
    \begin{tabular}{||c c||} 
     \hline
     Giá trị ban đầu & Giá trị thay thế \\
     \hline\hline
     1 day & 1 \\
     2 days & 2 \\
     3 days & 3 \\
     4 days & 4 \\
     5 days & 5 \\
     6 days & 6 \\
     1- 2 weeks & 11 \\
     2- 4 weeks & 21 \\
     4- 8 weeks & 42 \\
     8- 12 weeks & 84 \\
     12- 26 weeks & 133 \\
     26+ weeks & 182 \\
     \hline
    \end{tabular}
    \caption{LenghOfStay}
    \label{table:1}
\end{table}

\textbf{\underline{DSFS}}: Dữ liệu được chuyển thành số tháng và tính toán các giá trị thống kê min, max của từng bệnh nhân theo Y1, Y2 và Y12. Xem bảng sau:

\begin{table}[h!]
    \centering
    \begin{tabular}{||c c||} 
     \hline
     Giá trị ban đầu & Giá trị thay thế \\
     \hline\hline
     0- 1 month & 1 \\
     1- 2 months & 2 \\
     2- 3 months & 3 \\
     3- 4 months & 4 \\
     4- 5 months & 5 \\
     5- 6 months & 6 \\
     6- 7 months & 7 \\
     7- 8 months & 8 \\
     8- 9 months & 9 \\
     9-10 months & 10 \\
     10-11 months & 11 \\
     11-12 months & 12 \\
     \hline
    \end{tabular}
    \caption{DSFS}
    \label{table:1}
\end{table}

\textbf{\underline{CharlsonIndex}}: Dữ liệu được chuyển thành số và tính toán các giá trị thống kê min, max, avg của từng bệnh nhân theo Y1, Y2 và Y12. Xem bảng sau:

\begin{table}[h!]
    \centering
    \begin{tabular}{||c c||} 
     \hline
     Giá trị ban đầu & Giá trị thay thế \\
     \hline\hline
     0 & 0 \\
     1-2 & 2 \\
     3-4 & 4 \\
     5+ & 6 \\
     \hline
    \end{tabular}
    \caption{CharlsonIndex}
    \label{table:1}
\end{table}

\textbf{\underline{PrimaryConditionGroup}}: Đếm số lượng mỗi giá trị của PrimaryConditionGroup cho từng bệnh nhân theo Y1, Y2 và Y12.

\textbf{\underline{ProcedureGroup}}: Đếm số lượng mỗi giá trị của ProcedureGroup cho từng bệnh nhân theo Y1, Y2 và Y12.

\subsection{Bảng LabCount}

\textbf{\underline{LabCount}}: Chuyển giá trị 10+ thành 10 và tính toán các giá trị thống kê như min, max, avg, std, sum và số lượng claim (LabClaimCount) của từng bệnh nhân theo Y1, Y2 và Y12.

\subsection{Bảng DrugCount}

\textbf{\underline{DrugCount}}: Chuyển giá trị 7+ thành 7 và tính toán các giá trị thống kê như min, max, avg, std, sum và số lượng claim (LabClaimCount) của từng bệnh nhân theo Y1, Y2 và Y12.

\section{Dataset building}

Dữ liệu ban đầu sau khi tiền xử lý sẽ được dùng để xây dựng dữ liệu đầu vào theo 2 chiến lược bên dưới.

\subsection{Strategy A}

Ở chiến lược này, ta chỉ sử dụng dữ liệu một năm để xây dựng mô hình dự đoán kết quả của năm tiếp theo. Vì vậy tập dữ liệu được phân chia như bảng bên dưới:

\begin{table}[h!]
    \centering
    \begin{tabular}{|c c|} 
     \hline
     Training set & Testing set \\
     \hline\hline
     Claims Y1 & Claims Y2 \\
     Outcome: DaysInHospital Y2 &  Outcome: DaysInHospital Y3\\
     \hline
    \end{tabular}
    \caption{Training set và testing set cho Strategy A}
    \label{table:1}
\end{table}

Các feature trong Strategy A:

\begin{longtable}[c]{| c | c |}
     \hline
     Feature & Description \\
     \hline
     MemberID & Chỉ dùng làm index \\
     AgeAtFirstClaim & Tuổi lúc claim lần đầu \\
     Sex & Giới tính \\
     ClaimCount & Số lượng claim trong năm \\
     ProviderCount & Số lượng Provider duy nhất trong năm \\
     VendorCount & Số lượng Vendor duy nhất trong năm \\
     PCPCount & Số lượng PCP duy nhất trong năm \\
     SpecialtyCount\_<value> & Số lượng claim từng giá trị trong năm \\
     PlaceSvcCount\_<value> & Số lượng claim từng giá trị trong năm \\
     PayDelayMin & Min PayDelay trong năm \\
     PayDelayMax & Max PayDelay trong năm \\
     PayDelayAvg & Avg PayDelay trong năm \\
     PayDelayStd & Std PayDelay trong năm \\
     PayDelaySum & Sum PayDelay trong năm \\
     LengthOfStayMin & Min LengthOfStay trong năm \\
     LengthOfStayMax & Max LengthOfStay trong năm \\
     LengthOfStayAvg & Avg LengthOfStay trong năm \\
     LengthOfStayStd & Std LengthOfStay trong năm \\
     LengthOfStaySum & Sum LengthOfStay trong năm \\
     LengthOfStayCountNan & Số lượng LengthOfStay là NaN trong năm \\ 
     DSFSMin & Min DSFS trong năm \\
     DSFSMax & Max DSFS trong năm \\
     CharlsonIndexMin & Min CharlsonIndex trong năm \\
     CharlsonIndexMax & Max CharlsonIndex trong năm \\
     CharlsonIndexAvg & Avg CharlsonIndex trong năm \\
     PrimaryConditionGroupCount\_<value> & Số lượng claim từng giá trị trong năm \\
     ProcedureGroupCount\_<value> & Số lượng claim từng giá trị trong năm \\
     LabCountMax & Max LabCount trong năm \\
     LabCountMin & Min LabCount trong năm \\
     LabCountAvg & Avg LabCount trong năm \\
     LabCountStd & Std LabCount trong năm \\
     LabCountSum & Sum LabCount trong năm \\
     LabClaimCount & Số lượng claim của LabCount trong năm \\
     DrugCountMax & Max DrugCount trong năm \\
     DrugCountMin & Min DrugCount trong năm \\
     DrugCountAvg & Avg DrugCount trong năm \\
     DrugCountStd & Std DrugCount trong năm \\
     DrugCountSum & Sum DrugCount trong năm \\
     DrugClaimCount & Số lượng claim của DrugCount trong năm \\
     \hline
\end{longtable}


\subsection{Strategy B}

Trong chiến lược này, ta sẽ kết hợp các thuộc tính ở từng năm thành một bộ dữ liệu duy nhất. Sau đó sẽ chia tập training và testing theo tỉ lệ 8:2.

Các feature trong Strategy B:

\begin{longtable}[c]{| c | c |}
    \hline
    Feature & Description \\
    \hline
    MemberID & Chỉ dùng làm index \\
    AgeAtFirstClaim & Tuổi lúc claim lần đầu \\
    Sex & Giới tính \\
    ClaimCount & Số lượng claim toàn lịch sử \\
    ClaimCountLatestYear & Số lượng claim Y2 \\
    ProviderCount & Số lượng Provider duy nhất toàn lịch sử \\
    ProviderCountLatestYear & Số lượng Provider duy nhất trong Y2 \\
    VendorCount & Số lượng Vendor duy nhất toàn lịch sử \\
    VendorCountLatestYear & Số lượng Vendor duy nhất trong Y2 \\
    PCPCount & Số lượng PCP duy nhất toàn lịch sử \\
    PCPCountLatestYear & Số lượng PCP duy nhất trong Y2 \\
    SpecialtyCount\_<value> & Số lượng claim từng giá trị toàn lịch sử \\
    PlaceSvcCount\_<value> & Số lượng claim từng giá trị toàn lịch sử \\
    PayDelayMin & Min PayDelay toàn lịch sử \\
    PayDelayMax & Max PayDelay toàn lịch sử \\
    PayDelayAvg & Avg PayDelay toàn lịch sử \\
    PayDelayStd & Std PayDelay toàn lịch sử \\
    PayDelaySum & Sum PayDelay toàn lịch sử \\
    PayDelayMinLatestYear & Min PayDelay trong Y2 \\
    PayDelayMaxLatestYear & Max PayDelay trong Y2 \\
    PayDelayAvgLatestYear & Avg PayDelay trong Y2 \\
    PayDelayStdLatestYear & Std PayDelay trong Y2 \\
    PayDelaySumLatestYear & Sum PayDelay trong Y2 \\
    LengthOfStayMin & Min LengthOfStay toàn lịch sử \\
    LengthOfStayMax & Max LengthOfStay toàn lịch sử \\
    LengthOfStayAvg & Avg LengthOfStay toàn lịch sử \\
    LengthOfStayStd & Std LengthOfStay toàn lịch sử \\
    LengthOfStaySum & Sum LengthOfStay toàn lịch sử \\
    LengthOfStayCountNan & Số lượng LOS là NaN toàn lịch sử \\ 
    LengthOfStaySumLatestYear & Sum LOS trong Y2 \\
    LengthOfStayMaxLatestYear & Max LOS trong Y2 \\
    LengthOfStayCountNanLatestYear & Số lượng LengthOfStay là NaN trong Y2 \\ 
    DSFSMin & Min DSFS toàn lịch sử \\
    DSFSMax & Max DSFS toàn lịch sử \\
    DSFSMaxLatestYear & Max DSFS trong Y2 \\
    CharlsonIndexMin & Min CharlsonIndex toàn lịch sử \\
    CharlsonIndexMax & Max CharlsonIndex toàn lịch sử \\
    CharlsonIndexAvg & Avg CharlsonIndex toàn lịch sử \\
    CharlsonIndexMaxLatestYear & Max CharlsonIndex trong Y2 \\
    PrimaryConditionGroupCount\_<value> & Số lượng claim từng giá trị toàn lịch sử \\
    ProcedureGroupCount\_<value> & Số lượng claim từng giá trị toàn lịch sử \\
    LabCountSum & Sum LabCount toàn lịch sử \\
    LabClaimCount & Số lượng claim của LabCount toàn lịch sử \\
    LabCountSumLatestYear & Sum LabCount trong Y2 \\
    LabClaimCountLatestYear & Số lượng claim của LabCount trong Y2 \\
    DrugCountSum & Sum DrugCount toàn lịch sử \\
    DrugClaimCount & Số lượng claim của DrugCount toàn lịch sử \\
    DrugCountSumLatestYear & Sum DrugCount trong Y2 \\
    DrugClaimCountLatestYear & Số lượng claim của DrugCount trong Y2 \\
    \hline
\end{longtable}